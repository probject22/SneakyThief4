%Write MapGenerator explanation

In this simulator, the world consists of a map with two main types of grids, namely; movable and Immovable grids. Movable grids consist of pathways ,doors, target and shade area. Where the immovable grids only involve inner and outer walls. The following section explains the methods used to generate the maps used in the experiments.
Every method starts by placing the outer-walls; walls that surround the whole map and that do not hold doors nor windows. Next they generate all the other walls and the doors and windows on these walls. Consequently, the movable objects are placed on the map at the desired places.

\subsubsection{SplitRoom}
The $SplitRoom$ algorithm takes the full map and splits it in two at an arbitrary point, so two separate rooms are created. It takes the created rooms and splits them as well. It continues splitting the small rooms until either the maximum amount of splits is reached. Accordingly, it takes all the rooms and places two doors on each wall, hence, the agents will always find ways to leave and enter every room in multiple ways.
There $SplitRoom$ algorithm is created in two different versions, the original $SplitRoom$ and the $RandomSplitRoom$ algorithm. The original $SplitRoom$, splits alternating horizontally and vertically. The $RandomSplitRoom$ algorithm splits the rooms with a random orientation.

\subsubsection{URoom}
The $URoom$ algorithm places U-shaped walls on random locations and with random rotations on the map, until the set amount of U-shaped walls is reached.
The $StaticURoom$ algorithm divides the whole map into subsections and places U-shaped walls of the same sizes but different rotations in these subsections.
As the implementation of these algorithms does not allow closed rooms, the placement of doors is not implemented in this method.

\subsubsection{Maze}
