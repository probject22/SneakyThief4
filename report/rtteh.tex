
			The RTTE-h algorithm uses various geometrical features of the obstacles that the agent encounters to determine the best moving direction as an angle, and also the utilities of all discretised moving directions.

			%
			% Explain how RTTEh works
			%
			\begin{algorithm}[h!]
				\KwData{The agent's current cell $s$}
				\KwResult{Utilities of neighbours of $s$}
				Mark all moving directions of $s$ as open\;
				Propagate rays over the edges of the moving directions\;
				\For{each ray $r$ hitting an obstacle}{
					Let $o$ be the obstacle hit by $r$\;
					Extract border $b$ from $o$\;
					Detect closed directions of cell $s$ using $b$\;
					Extract geometric features of $o$\;
					Determine the best moving direction to avoid $o$\;
				}
				Merge the results (see alg.~\ref{alg:rrtehmerge})\;
				\Return utilities\;
				\label{alg:rtteh}
				\caption{RTTE-h algorithm~\protect\cite{undeger2010multi}}
			\end{algorithm}

			The merging phase of the algorithm determines the utilities as is described in alg.~2.


			\begin{algorithm}[h!]
				\KwData{current coordinate of the guard $s$; current coordinate of the target $t$; results from the rays $r$.}
				\If{All neighbours of $s$ are closed}{
					\Return failure\;
				}
				Determine most constraining obstacle $m$ from $r$\;
				\eIf{$m$ exists}{
					Let proposed direction $d$ be the direction that gets around $m$\;
				}{
					Let proposed direction $d$ be direct direction to $t$\;
				}
				\For{each neighbour $n$ of $s$}{
					\eIf{$n$ is closed}{
						Let utility of $n$ be $0$\;
					}{
						Let $dif$ be the smallest angle between $d$ and the direction of the cell $n$\;
						Let the utility of $n$ be $(181-dif)/181$\;
					}
				}
				\label{alg:rrtehmerge}
				\caption{Merging Phase of RTTE-h~\protect\cite{undeger2010multi}}
				\Return utilities\;
			\end{algorithm}

			The RTTEh alg.~6 provides answers to two shortcomings of the A* algorithm for path search. First, it is a continuous space algorithm, so it is not dependant on any kind of discretisation for real world applications. Second, it is capable of identifying whether a certain coordinate in space is unreachable without having to explore a large amount of possible routes.

			Another property that will be useful later on is that this algorithm returns an ordering of all moving directions, instead of just the best one.

			Although the algorithm operates in continuous space, the moving directions are still discretised. In real world applications this is usually not the case, and this will therefore lead to a suboptimal result.\cite{undeger2009real, undeger2010multi, undeger2007single}.