\documentclass{report}

\usepackage[toc,page]{appendix}
\usepackage[british]{babel}
\usepackage[affil-it]{authblk}
\usepackage{apacite}

\title{Sneaky Thief}
\author{Stan Kerstjens\thanks{i6048794 : \texttt{s.kerstjens@student.maastrichtuniversity.nl}}}
\author{ing. Robert Stevens\thanks{}} % Sorry if I didn't get all your credentials - Stan
\author{Sina \thanks{}} %Sorry I do not dare spelling your last name - Stan
\author{Rob Clinch\thanks{}}
\author{Sharon Hallmanns\thanks{}} % is this spelling correct? - Stan
\affil{Department of Knowledge Engineering, Maastricht University}

\date{\today}
\begin{document}
\maketitle
\pagenumbering{roman}
\section*{Preface}
\section*{Abstract}
\section*{List of Graphs and Figures}
\section*{List of Abbreviations and Symbols}
\tableofcontents\newpage



\newpage
\begin{center} This page is intentionally left blank \end{center}
\pagenumbering{arabic}
\newpage

%Begin of report body
\chapter{Introduction}
	The problem posed can be divided into three main problems. The first problem is that guards put into a unknown environment should be able to create a 'mental map' of the environment. This is what we will call the exploration task. The second problem is that the guards should monitor the entire area as closely as possible to make it as difficult as possible for a possible intruder to pass through the area unseen. This will be called the coverage task. The third problem is that once an intruder has been detected, the guards should be able to actually catch the intruder. This will be called the pursuit task.

	In every of these three main problems the guard should figure out two things, namely where it should go, and how it gets there. The 'where'-question is largely dependent on the specific task at hand, e.g. even if an intruder would only be 1 unit of distance away, but the task is exploration, then the guard does not necessarily have to move towards the intruder.

	The 'how'-question, on the other hand, is in many cases independent of the exact task that we want to achieve, and therefore finding the shortest route to the destination does usually suffice, and is often even the best route. We will call the 'how'-question the pathfinding problem.

	The final task of the guards is to combine the three main problems and decide when the guard should solve what problem. The intuitive solution would be to explore untill the entire environment is known, and then start, and continue, covering the area. During this process pursuit is started as soon as an intruder is detected. This intuitive answer, however, runs into issues once it is not known to the guard how big the area actually is, so it will never know if the map if fully explored or not. Another possible issue arises when an intruder, for instance, dissapears behind a wall. The pursuing guard does not detect the intruder anymore and stops pursuit. He should, however, have the intelligence to predict where the guard has gone.

\chapter{Approaches}
	
	\section{Path finding}
		As already mentioned in the introduction, the answer to the question on what path to take to a certain location is in many cases not important as long as we find the shortest path. This is why we can first discuss general methods of path finding without refering to what goal the moving of the agents serves. There are also cases where the shortest path is not the best path. This is heavily dependent on the task, and will therefore be discussed in the sections dealing with the specific tasks.

		\subsection{A-star}
			The A-star algorithm is a general purpose algorithm that searches a graph and is capable of providing the optimal path from one node to another. A-star is a heuristic search, and like many heuristic search algorithms its effectiveness depends on the heuristic employed. In our simulations the nodes of the graphs will be simple coordinates, so for our heuristic function we can simply take the direct distance to the destination coordinate as our heuristic function.

			%
			% Explain how A-star works
			%

			A great disadvantage of the A-star algorithm in path finding is that it will not identify a certain coordinate as unreachable untill all possible routes have failed. As this would in take to long in most simulations to compute, it is necessary to provide a maximum amount of nodes the algorithm is allowed to explore before it has to give up the search. This, of course, is at the risk of falsly concluding that a certain position is unreachable.

			Another disadvantage of the A-star algorithm is that it is necessary to discretise the world in order for the algorithm to run. In real-world applications for robotic systems, most robots will not find themselves in a discrete world. Therefore the A-star algorithm will be very unlikely to find the optimal path in continuous world applications. If, however, the discretization is sufficiently narrow the proposed path will probably be sufficient for most purposes. Unfortunately, narrowing the discretisation inherently leads to the need for more computing power.

		\subsection{RTTEh}

			%
			% Explain how RTTEh works
			%

			The RTTEh algorithm provides answers to two shortcomings of the A-star algorithm for path search. First, it is a continuous space algorithm, so it is not dependend on any kind of discretisation for real world applications. Second, it is capable of identifying whether a certain coordinate in space is unreachable without having to explore a large amount of possible routes.

		\subsection{MTES}
		

	\section{Exploration}

	\section{Coverage}

	\section{Pursuit}
		For persuit there is a very simple intuitive answer, namely to go directly to the intruder following the shortest path. This, however, will only work if either the intruder is not actively avoiding the guard, or the guard is strictly faster than the intruder. As we are not prepared to assume any of these conditions, we need to act slightly more intelligently.

		\subsection{Blocking Escape Directions}
			A possible way of catching an intruder that is faster than the guards is to use the fact that there multiple guards persuing a single intruder. In this case the guards should surround the intruder as much as possible, as to block all the possible direction in which he could escape. Then the guards can draw in and catch the intruder. The most naive way of doing this is to disperse the available guards equally amongst the circumference of an intruder. The guard closest to the intruder will always move directly towards the intruder following the shortest path. The other guards will establish the amount of pursuing guards ($n$) and each block one of the angles resulting from dividing the total angle in which the agent can move in equal parts ($2\pi / n$).

			%
			% Insert explanatory graphics
			%


\chapter{Methods}
	%This chapter should describe what experiments we perform and how.

\chapter{Implementation}
	%This chapter should describe our implementation of the software

\chapter{Results}
	% This chapter should contain all experimental results that we obtain (lots of graphs!)


\chapter{Conclusions}
	% This chapter should contain all our conclusions of how the experimental results compare to our initial expectations.
	% Also it should contain our evaluation of the different approaches we took towards the problems

\bibliographystyle{apacite}
\bibliography{references.bib}
% All references APA style

\begin{appendices}
\chapter{Software Engineering Patterns and Principles}
	\section{UML diagrams}
\chapter{Other Data}
	
\end{appendices}



\end{document}
